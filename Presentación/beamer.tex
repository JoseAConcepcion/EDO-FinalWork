\documentclass{beamer}
\usefonttheme[onlylarge]{structuresmallcapsserif}
\usefonttheme[onlysmall]{structurebold}
\usetheme{Warsaw}
\setbeamercovered{transparent}
%-----------------------------------------Paquetes y comandos personales-------------------------------%
\usepackage{fancybox,color,tcolorbox}
\usepackage{verbatim}
\usepackage[english,spanish,activeacute]{babel}
%\usepackage[latin1]{inputenc}
\usepackage{inputenc}
\usepackage{latexsym}
\usepackage{amsmath}
\usepackage{graphicx} % Allows including images
\usepackage{booktabs}
\usepackage{amssymb}
\usepackage{multimedia}
\usepackage{pifont,pgfcore}
\usepackage{tcolorbox}
\usepackage{animate}
\usepackage{ucs}
%-----------------------------------------------------Nuevos Ambientes----------------------------------%

\newtheorem*{Theorem*}{Teorema}
\newtheorem*{Definition*}{Definici\'on}
\newtheorem*{Example*}{Ejemplo}

%-----------------------------------------------------Nuevos Comandos-----------------------------------%
%-----------------------------------------------------Titulo--------------------------------------------%

\vspace*{-0.8cm}
\title[T\'itulo del trabajo]{T\'itulo de la presentaci\'on del trabajo}
\author[Machado, Toledo, Moreno, Concepci\'on, Navarro]
		{
			Daniel Machado \\
			Daniel Toledo \\
			Osvaldo Moreno \\
			Jos\'e A. Concepci\'on \\
			Adri\'an Navarro \\
			{\small C211}
		}
			
\date{Mayo 2023}

\begin{document}
%-----Redefiniendo colores----------%
\definecolor{green}{rgb}{0.1,0.5,0.3}
\definecolor{vio}{rgb}{0.8,0.5,0.2}
\definecolor{g}{rgb}{0.93,0.93,0.93}
\begin{frame}
    
    \titlepage
    \vspace*{-0.65cm}
    \begin{center}
        \colorbox{black}{\textbf{\begin{large}\textcolor{white}{Defensa del proyecto final.}\end{large}}}\\\ \\
        \vspace*{-0.3cm}
        \scriptsize \textbf{\textcolor{blue}{Facultad Matem\'atica Computaci\'on}}\\
        \textbf{\textcolor{blue}{Universidad de La Habana}}
    \end{center}
\end{frame}
%-----------------------------------------------

%-------------------------------------------------
\begin{frame}
	\frametitle{Temas a tratar} 
	\tableofcontents 
\end{frame}
%------------------------------------

%-------------------------------------
\section{Introducci\'on}
\begin{frame}
\begin{minipage}{10cm}
	El artículo titulado "Local Analysis of the Prey-Predator Model with Stage-Structure Prey and Holling
	Type Functional Responses" por Dian Savitri analiza el modelo presa-depredador utilizando respuestas funcionales de Holling de tipo II para presas adultas y tipo I para presas j\'ovenes.
\end{minipage}
\end{frame}

\begin{frame}
\frametitle{Objetivos}
	\begin{minipage}{10cm}
		\begin{itemize}
			\item Determinar los puntos de equilibrio del modelo.
			\item Analizar la estabilidad local de los puntos de equilibrio utilizando la matriz jacobiana y valores propios.
			\item Observar el comportamiento dinámico del modelo mediante
			simulaciones numéricas y diagramas de fase.
		\end{itemize}
	\end{minipage}
\end{frame}

\begin{frame}
	\frametitle{T\'ecnicas utilizadas}
	\begin{minipage}{10cm}
		\begin{itemize}
			\item Análisis matemático para determinar los puntos de equilibrio y condiciones de estabilidad.
			\item Cálculo de la matriz jacobiana y valores propios en cada punto de equilibrio.
			\item Aplicación del Criterio de Routh-Hurwitz
			para analizar la estabilidad del equilibrio interior.
			\item Simulaciones numéricas utilizando el método Runge-Kutta de
			cuarto orden.
			\item Análisis de bifurcación para estudiar la posible existencia de ciclos límite.
		\end{itemize} 
	\end{minipage}
\end{frame}	

\section{Desarrollo}
\begin{frame}
\frametitle{Modelo Propuesto}
\begin{minipage}{10cm}
	El artículo analiza previamente tres modelos matemáticos que abordan las interacciones presa-depredador con diferentes enfoques:
	\begin{itemize}
		\item El modelo propuesto por Huenchucona considera dos presas y un depredador, utilizando una función de respuesta de tipo Beddington-DeAngelis.
		\item  El modelo de Savitri y Abadi tiene en cuenta la estructura 
		por etapas de la presa, utilizando funciones de respuesta diferentes para presas inmaduras y maduras.
		\item El modelo de Castellanos y Chan-López estudia una cadena trófica de tres niveles, donde la presencia del depredador superior afecta la estabilidad del sistema.
	\end{itemize}
\end{minipage}
\end{frame}

\begin{frame}
	\frametitle{Modelo Propuesto}
	\begin{minipage}{10cm}
	Estos modelos constituyen una base para el análisis 
	del sistema de ecuaciones diferenciales propuesto donde se definen \emph{x(t)}, \emph{y(t)} y \emph{z(t)} como la densidad de la poblaci\'on de presas j\'ovenes, presas adultas y depredadores respectivamente.
	
	\begin{equation} \label{twoPreyonePredatorEDO}
	\begin{gathered}
	\frac{d x}{d t}=r x\left(1-\frac{x}{k}\right)-\beta x-\alpha x z \\
	\frac{d y}{d t}=\beta x-\frac{\eta y z}{y+m}-\mu y \\
	\frac{d z}{d t}=\alpha_1 x z + \rho z^2-\frac{\eta_1 z^2}{y+m}
	\end{gathered}
	\end{equation}
		
	\end{minipage}
\end{frame}



\begin{frame}
	\frametitle{Respuestas funcionales de Holling}
	\begin{alertblock}{\begin{dinglist}{45}
				\item Holling Tipo I
		\end{dinglist}}
		\begin{minipage}{10.5cm}
			Expresa la existencia de un aumento lineal en la tasa de captura del depredador con respecto a la densidad de presas, hasta alcanzar el valor en 
			el que la tasa máxima de depredación permanece constante. Esta respuesta funcional es una función continua por partes, con una sección lineal y una sección constante.
		\end{minipage}
	\end{alertblock}
\begin{alertblock}{\begin{dinglist}{45}
			\item Holling Tipo II
	\end{dinglist}}
	\begin{minipage}{10.5cm}
		Expresa un aumento en el consumo que se desacelera a medida que aumentan las presas consumidas, alcanzando la tasa máxima de consumo de 
		depredadores de forma asintótica. La respuesta funcional de Holling tipo II es monotóna creciente, lo que significa que la tasa de consumo 
		de los depredadores aumenta con la densidad de presas.
	\end{minipage}
\end{alertblock}
\end{frame}



\begin{frame}
\frametitle{Motivaci\'on}
\begin{minipage}{10cm}
Esto resulta algo lento y complicado, en la mayor\'ia de los problemas, raz\'on por la cual, en la pr\'actica casi no se utilizan. El m\'etodo de Euler, lamentablemente requiere de un paso muy pequeño para una precisi\'on razonable. 
\end{minipage}
\end{frame}	

\begin{frame}
	\frametitle{Modelado matem\'atico }
\begin{alertblock}{\begin{dinglist}{45}
			\item Postulados
	\end{dinglist}}
    \begin{minipage}{10cm}
	Los m\'etodos de Runge-Kutta tienen el error local de truncamiento del mismo orden que los m\'etodos de Taylor, pero prescinden del c\'alculo y evaluaci\'on de las derivadas de la funci\'on f
     \end{minipage}
\end{alertblock}
\end{frame}

\section{Formulaci\'on matem\'atica.}
%----------------------------------
\begin{frame}
	\frametitle{Conclusiones}
	\begin{minipage}{10cm}
	Se ha demostrado para el sistema 
	\end{minipage}
\end{frame}

\begin{frame}
	\frametitle{Recomendaciones}
	\begin{minipage}{10cm}
		Se puede extender los resultados a modelos... 
	\end{minipage}
\end{frame}
%----------------------------------
\section{Bibliograf\'ia del tema}
\begin{frame}
\frametitle{Bibliograf\'ia}
\begin{thebibliography}{99}

\bibitem[Sanjuan,2016]{sanjuan2016} M. A. Fern\'andez Sanju\'an (2016). Din\'amica No Lineal, Teor\'ia del Caos y Sistemas Complejos: una perspectiva hist\'orica. {\em Rev. R. Acad. Cienc. Exact. F\'is .Nat.} \textbf{Vol}. 109, N. 1?2, pp. 107-126.
\end{thebibliography}	
\end{frame}

\end{document}

