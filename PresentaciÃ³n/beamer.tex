\documentclass{beamer}
\usefonttheme[onlylarge]{structuresmallcapsserif}
\usefonttheme[onlysmall]{structurebold}
\usetheme{Warsaw}
\setbeamercovered{transparent}
%-----------------------------------------Paquetes y comandos personales-------------------------------%
\usepackage{fancybox,color,tcolorbox}
\usepackage{verbatim}
\usepackage[english,spanish,activeacute]{babel}
%\usepackage[latin1]{inputenc}
\usepackage{inputenc}
\usepackage{latexsym}
\usepackage{amsmath}
\usepackage{graphicx} % Allows including images
\usepackage{booktabs}
\usepackage{amssymb}
\usepackage{multimedia}
\usepackage{pifont,pgfcore}
\usepackage{tcolorbox}
\usepackage{animate}
\usepackage{ucs}
%-----------------------------------------------------Nuevos Ambientes----------------------------------%

\newtheorem*{Theorem*}{Teorema}
\newtheorem*{Definition*}{Definici\'on}
\newtheorem*{Example*}{Ejemplo}

%-----------------------------------------------------Nuevos Comandos-----------------------------------%
%-----------------------------------------------------Titulo--------------------------------------------%

\vspace*{-0.8cm}
\title[T\'itulo del trabajo]{T\'itulo de la presentaci\'on del trabajo}
\author[Machado, Toledo, Moreno, Concepci\'on, Navarro]
		{
			Daniel Machado \\
			Daniel Toledo \\
			Osvaldo Moreno \\
			Jos\'e A. Concepci\'on \\
			Adri\'an Navarro \\
			{\small C211}
		}
			
\date{Mayo 2023}

\begin{document}
%-----Redefiniendo colores----------%
\definecolor{green}{rgb}{0.1,0.5,0.3}
\definecolor{vio}{rgb}{0.8,0.5,0.2}
\definecolor{g}{rgb}{0.93,0.93,0.93}
\begin{frame}
    
    \titlepage
    \vspace*{-0.65cm}
    \begin{center}
        \colorbox{black}{\textbf{\begin{large}\textcolor{white}{Defensa del proyecto final.}\end{large}}}\\\ \\
        \vspace*{-0.3cm}
        \scriptsize \textbf{\textcolor{blue}{Facultad Matem\'atica Computaci\'on}}\\
        \textbf{\textcolor{blue}{Universidad de La Habana}}
    \end{center}
\end{frame}
%-----------------------------------------------

%-------------------------------------------------
\begin{frame}
	\frametitle{Temas a tratar} 
	\tableofcontents 
\end{frame}
%------------------------------------

%-------------------------------------
\section{Introducci\'on}
\begin{frame}
\begin{minipage}{10cm}
	Los m\'etodos de Taylor tienen la propiedad de un error local de truncamiento de orden superior, pero la desventaja de requerir el c\'alculo y la evaluaci\'on de las derivadas de f.
\end{minipage}
\end{frame}

\section{Desarrollo}
\begin{frame}
\begin{minipage}{10cm}
	Aqui iria el comienzo del desarrollo con el tema a tratar
\end{minipage}
\end{frame}

\begin{frame}
\frametitle{Motivaci\'on}
\begin{minipage}{10cm}
Esto resulta algo lento y complicado, en la mayor\'ia de los problemas, raz\'on por la cual, en la pr\'actica casi no se utilizan. El m\'etodo de Euler, lamentablemente requiere de un paso muy pequeño para una precisi\'on razonable. 
\end{minipage}
\end{frame}	

\begin{frame}
	\frametitle{Modelado matem\'atico }
\begin{alertblock}{\begin{dinglist}{45}
			\item Postulados
	\end{dinglist}}
    \begin{minipage}{10cm}
	Los m\'etodos de Runge-Kutta tienen el error local de truncamiento del mismo orden que los m\'etodos de Taylor, pero prescinden del c\'alculo y evaluaci\'on de las derivadas de la funci\'on f
     \end{minipage}
\end{alertblock}
\end{frame}

\section{Formulaci\'on matem\'atica.}
%----------------------------------
\begin{frame}
	\frametitle{Conclusiones}
	\begin{minipage}{10cm}
	Se ha demostrado para el sistema 
	\end{minipage}
\end{frame}

\begin{frame}
	\frametitle{Recomendaciones}
	\begin{minipage}{10cm}
		Se puede extender los resultados a modelos... 
	\end{minipage}
\end{frame}
%----------------------------------
\section{Bibliograf\'ia del tema}
\begin{frame}
\frametitle{Bibliograf\'ia}
\begin{thebibliography}{99}

\bibitem[Sanjuan,2016]{sanjuan2016} M. A. Fern\'andez Sanju\'an (2016). Din\'amica No Lineal, Teor\'ia del Caos y Sistemas Complejos: una perspectiva hist\'orica. {\em Rev. R. Acad. Cienc. Exact. F\'is .Nat.} \textbf{Vol}. 109, N. 1?2, pp. 107-126.
\end{thebibliography}	
\end{frame}

\end{document}

